\documentclass[10pt,]{article}
\usepackage{lmodern}
\usepackage{setspace}
\setstretch{1.5}
\usepackage{amssymb,amsmath}
\usepackage{ifxetex,ifluatex}
\usepackage{fixltx2e} % provides \textsubscript
\ifnum 0\ifxetex 1\fi\ifluatex 1\fi=0 % if pdftex
  \usepackage[T1]{fontenc}
  \usepackage[utf8]{inputenc}
\else % if luatex or xelatex
  \ifxetex
    \usepackage{mathspec}
  \else
    \usepackage{fontspec}
  \fi
  \defaultfontfeatures{Ligatures=TeX,Scale=MatchLowercase}
\fi
% use upquote if available, for straight quotes in verbatim environments
\IfFileExists{upquote.sty}{\usepackage{upquote}}{}
% use microtype if available
\IfFileExists{microtype.sty}{%
\usepackage{microtype}
\UseMicrotypeSet[protrusion]{basicmath} % disable protrusion for tt fonts
}{}
\usepackage[margin=2.5cm]{geometry}
\usepackage{hyperref}
\hypersetup{unicode=true,
            pdftitle={Title},
            pdfauthor={Alex Phillips},
            pdfborder={0 0 0},
            breaklinks=true}
\urlstyle{same}  % don't use monospace font for urls
\usepackage{color}
\usepackage{fancyvrb}
\newcommand{\VerbBar}{|}
\newcommand{\VERB}{\Verb[commandchars=\\\{\}]}
\DefineVerbatimEnvironment{Highlighting}{Verbatim}{commandchars=\\\{\}}
% Add ',fontsize=\small' for more characters per line
\newenvironment{Shaded}{}{}
\newcommand{\KeywordTok}[1]{\textcolor[rgb]{0.00,0.44,0.13}{\textbf{{#1}}}}
\newcommand{\DataTypeTok}[1]{\textcolor[rgb]{0.56,0.13,0.00}{{#1}}}
\newcommand{\DecValTok}[1]{\textcolor[rgb]{0.25,0.63,0.44}{{#1}}}
\newcommand{\BaseNTok}[1]{\textcolor[rgb]{0.25,0.63,0.44}{{#1}}}
\newcommand{\FloatTok}[1]{\textcolor[rgb]{0.25,0.63,0.44}{{#1}}}
\newcommand{\ConstantTok}[1]{\textcolor[rgb]{0.53,0.00,0.00}{{#1}}}
\newcommand{\CharTok}[1]{\textcolor[rgb]{0.25,0.44,0.63}{{#1}}}
\newcommand{\SpecialCharTok}[1]{\textcolor[rgb]{0.25,0.44,0.63}{{#1}}}
\newcommand{\StringTok}[1]{\textcolor[rgb]{0.25,0.44,0.63}{{#1}}}
\newcommand{\VerbatimStringTok}[1]{\textcolor[rgb]{0.25,0.44,0.63}{{#1}}}
\newcommand{\SpecialStringTok}[1]{\textcolor[rgb]{0.73,0.40,0.53}{{#1}}}
\newcommand{\ImportTok}[1]{{#1}}
\newcommand{\CommentTok}[1]{\textcolor[rgb]{0.38,0.63,0.69}{\textit{{#1}}}}
\newcommand{\DocumentationTok}[1]{\textcolor[rgb]{0.73,0.13,0.13}{\textit{{#1}}}}
\newcommand{\AnnotationTok}[1]{\textcolor[rgb]{0.38,0.63,0.69}{\textbf{\textit{{#1}}}}}
\newcommand{\CommentVarTok}[1]{\textcolor[rgb]{0.38,0.63,0.69}{\textbf{\textit{{#1}}}}}
\newcommand{\OtherTok}[1]{\textcolor[rgb]{0.00,0.44,0.13}{{#1}}}
\newcommand{\FunctionTok}[1]{\textcolor[rgb]{0.02,0.16,0.49}{{#1}}}
\newcommand{\VariableTok}[1]{\textcolor[rgb]{0.10,0.09,0.49}{{#1}}}
\newcommand{\ControlFlowTok}[1]{\textcolor[rgb]{0.00,0.44,0.13}{\textbf{{#1}}}}
\newcommand{\OperatorTok}[1]{\textcolor[rgb]{0.40,0.40,0.40}{{#1}}}
\newcommand{\BuiltInTok}[1]{{#1}}
\newcommand{\ExtensionTok}[1]{{#1}}
\newcommand{\PreprocessorTok}[1]{\textcolor[rgb]{0.74,0.48,0.00}{{#1}}}
\newcommand{\AttributeTok}[1]{\textcolor[rgb]{0.49,0.56,0.16}{{#1}}}
\newcommand{\RegionMarkerTok}[1]{{#1}}
\newcommand{\InformationTok}[1]{\textcolor[rgb]{0.38,0.63,0.69}{\textbf{\textit{{#1}}}}}
\newcommand{\WarningTok}[1]{\textcolor[rgb]{0.38,0.63,0.69}{\textbf{\textit{{#1}}}}}
\newcommand{\AlertTok}[1]{\textcolor[rgb]{1.00,0.00,0.00}{\textbf{{#1}}}}
\newcommand{\ErrorTok}[1]{\textcolor[rgb]{1.00,0.00,0.00}{\textbf{{#1}}}}
\newcommand{\NormalTok}[1]{{#1}}
\IfFileExists{parskip.sty}{%
\usepackage{parskip}
}{% else
\setlength{\parindent}{0pt}
\setlength{\parskip}{6pt plus 2pt minus 1pt}
}
\setlength{\emergencystretch}{3em}  % prevent overfull lines
\providecommand{\tightlist}{%
  \setlength{\itemsep}{0pt}\setlength{\parskip}{0pt}}
\setcounter{secnumdepth}{0}
% Redefines (sub)paragraphs to behave more like sections
\ifx\paragraph\undefined\else
\let\oldparagraph\paragraph
\renewcommand{\paragraph}[1]{\oldparagraph{#1}\mbox{}}
\fi
\ifx\subparagraph\undefined\else
\let\oldsubparagraph\subparagraph
\renewcommand{\subparagraph}[1]{\oldsubparagraph{#1}\mbox{}}
\fi

\title{Title}
\author{Alex Phillips}
\date{\today}

\begin{document}
\maketitle

Git is about version control. Changes made to files are called commits.

\begin{enumerate}
\def\labelenumi{\arabic{enumi}.}
\tightlist
\item
  To set git account locally
\end{enumerate}

\begin{Shaded}
\begin{Highlighting}[]
\NormalTok{$ }\KeywordTok{git} \NormalTok{config --global user.email }\StringTok{"user.email@email.com"}
\end{Highlighting}
\end{Shaded}

\begin{enumerate}
\def\labelenumi{\arabic{enumi}.}
\tightlist
\item
  To initialize folder as git repository.
\end{enumerate}

\begin{Shaded}
\begin{Highlighting}[]
\NormalTok{$ }\KeywordTok{git} \NormalTok{init}
\NormalTok{$ }\KeywordTok{git} \NormalTok{status}
\KeywordTok{On} \NormalTok{branch master}

\KeywordTok{Initial} \NormalTok{commit}

\KeywordTok{Untracked} \NormalTok{files:}
  \KeywordTok{(use} \StringTok{"git add <file>..."} \NormalTok{to include in what will be committed}\KeywordTok{)}

        \KeywordTok{.git_workshop_notes.md.swp}
        \KeywordTok{git_workshop_notes.md}
        \KeywordTok{git_workshop_notes.pdf}
        \KeywordTok{git_workshop_notes.tex}
        \KeywordTok{git_workshop_notestex}
        \KeywordTok{git_workshop_notestex.aux}
        \KeywordTok{git_workshop_notestex.log}
        \KeywordTok{git_workshop_notestex.out}
        \KeywordTok{git_workshop_notestex.pdf}

\KeywordTok{nothing} \NormalTok{added to commit but untracked files present (use }\StringTok{"git add"} \NormalTok{to track)}

\end{Highlighting}
\end{Shaded}

\begin{itemize}
\tightlist
\item
  By default, first commit goes to Master.
\item
  Here I have created files, and they are UNTRACKED!!
\item
  As soon as you make git aware of them, git will begin tracking their
  changes.
\item
  git add stages the changes.
\item
  Some people add each file separately, then commit in bulk.
\end{itemize}

\begin{Shaded}
\begin{Highlighting}[]
\NormalTok{$ }\KeywordTok{git} \NormalTok{add *}
\NormalTok{$ }\KeywordTok{git} \NormalTok{status}
\KeywordTok{On} \NormalTok{branch master}

\KeywordTok{Initial} \NormalTok{commit}

\KeywordTok{Changes} \NormalTok{to be committed:}
  \KeywordTok{(use} \StringTok{"git rm --cached <file>..."} \NormalTok{to unstage}\KeywordTok{)}

        \KeywordTok{new} \NormalTok{file:   git_workshop_notes.md}
        \KeywordTok{new} \NormalTok{file:   git_workshop_notes.pdf}
        \KeywordTok{new} \NormalTok{file:   git_workshop_notes.tex}
        \KeywordTok{new} \NormalTok{file:   git_workshop_notestex}
        \KeywordTok{new} \NormalTok{file:   git_workshop_notestex.aux}
        \KeywordTok{new} \NormalTok{file:   git_workshop_notestex.log}
        \KeywordTok{new} \NormalTok{file:   git_workshop_notestex.out}
        \KeywordTok{new} \NormalTok{file:   git_workshop_notestex.pdf}

\KeywordTok{Untracked} \NormalTok{files:}
  \KeywordTok{(use} \StringTok{"git add <file>..."} \NormalTok{to include in what will be committed}\KeywordTok{)}

        \KeywordTok{.git_workshop_notes.md.swp}
\end{Highlighting}
\end{Shaded}

\begin{itemize}
\item
  Now it's tracking them, but they have not been committed yet.
\item
  git commit

\begin{Shaded}
\begin{Highlighting}[]
\NormalTok{$ }\KeywordTok{git} \NormalTok{commit}
\NormalTok{[}\KeywordTok{opens} \NormalTok{vi editor and wants message with the stuff]}
\NormalTok{[}\KeywordTok{master} \NormalTok{(root-commit) }\KeywordTok{0e0d499}\NormalTok{] Start notes on this tutorial folder}
 \KeywordTok{8} \NormalTok{files changed, 1280 insertions(+)}
 \KeywordTok{create} \NormalTok{mode 100644 git_workshop_notes.md}
 \KeywordTok{create} \NormalTok{mode 100644 git_workshop_notes.pdf}
 \KeywordTok{create} \NormalTok{mode 100644 git_workshop_notes.tex}
 \KeywordTok{create} \NormalTok{mode 100644 git_workshop_notestex}
 \KeywordTok{create} \NormalTok{mode 100644 git_workshop_notestex.aux}
 \KeywordTok{create} \NormalTok{mode 100644 git_workshop_notestex.log}
 \KeywordTok{create} \NormalTok{mode 100644 git_workshop_notestex.out}
 \KeywordTok{create} \NormalTok{mode 100644 git_workshop_notestex.pdf}
\end{Highlighting}
\end{Shaded}
\item
  vi editor is default, but I prefer vim:

\begin{Shaded}
\begin{Highlighting}[]
\NormalTok{$  }\KeywordTok{git} \NormalTok{config --global user.editor }\StringTok{"vim"}
\end{Highlighting}
\end{Shaded}
\item
  git log tells use the history of changes that we have made. ```bash \$
  git log commit 0e0d499b750312cf8fe371799803dfdb5637d321 Author:
  alexprp
  \href{mailto:alexprphillips@gmail.com}{\nolinkurl{alexprphillips@gmail.com}}
  Date: Thu Feb 25 13:32:33 2016 -0600

  Start notes on this tutorial folder
\end{itemize}

\$ git log --oneline 0e0d499 Start notes on this tutorial folder ```

\begin{itemize}
\item
  This is why it's super important to leave good commit messages.
  Otherwise, the message in the log is meaningless.
\item
  I have made more changes to this notes file and its derivatives since
  we last committed. Let's see what that does ```bash \$ git status On
  branch master Changes not staged for commit: (use ``git add \ldots{}''
  to update what will be committed) (use ``git checkout -- \ldots{}'' to
  discard changes in working directory)

  modified: git\_workshop\_notes.md modified: git\_workshop\_notes.pdf
  modified: git\_workshop\_notes.tex
\end{itemize}

Untracked files: (use ``git add \ldots{}'' to include in what will be
committed)

\begin{verbatim}
.git_workshop_notes.md.swp
\end{verbatim}

no changes added to commit (use ``git add'' and/or ``git commit -a'')
\texttt{-\ git\ diff\ tells\ us\ the\ changes\ specifically\ that\ have\ been\ made.\ \ -\ I\ don\textquotesingle{}t\ put\ that\ here\ because\ it\textquotesingle{}s\ the\ same\ text\ and\ pandoc/latex/markdown\ get\ \ \ confused.}bash
\$ rm \emph{notestex} \$ git add * \$ git commit -m `deleted accidental
files and added more notes' {[}master 904dcea{]} deleted accidental
files and added more notes 3 files changed, 203 insertions(+), 3
deletions(-) rewrite git\_workshop\_notes.pdf (62\%) \$ git log

commit 904dceaff5ee685c3a52d11e673068e9c22baed0 Author: alexprp
\href{mailto:alexprphillips@gmail.com}{\nolinkurl{alexprphillips@gmail.com}}
Date: Thu Feb 25 13:58:12 2016 -0600

\begin{verbatim}
deleted accidental files and added more notes
\end{verbatim}

commit 0e0d499b750312cf8fe371799803dfdb5637d321 Author: alexprp
\href{mailto:alexprphillips@gmail.com}{\nolinkurl{alexprphillips@gmail.com}}
Date: Thu Feb 25 13:32:33 2016 -0600

\begin{verbatim}
Start notes on this tutorial folder
\end{verbatim}

\$ git log --oneline

904dcea deleted accidental files and added more notes 0e0d499 Start
notes on this tutorial folder ``` - `git --staged' only shows staged
chages

\subsection{Branches}\label{branches}

\begin{itemize}
\tightlist
\item
  Master always points to latest commit; it's a pointer
\item
  HEAD is a special pointer that points to Master, not directly to the
  latest commit. HEAD points to the \textbf{parent of the nest commit}.
\item
  If chages are made off of another instance of the previous commit, a
  new master is created. If there is a dangling Master, it will be
  deleted periodically, automatically by git if it has not been
  committed.
\end{itemize}

\subsection{Folders}\label{folders}

\begin{Shaded}
\begin{Highlighting}[]
\NormalTok{$ }\KeywordTok{mkdir} \NormalTok{folder1}

\NormalTok{$ }\KeywordTok{git} \NormalTok{add folder1}

\NormalTok{[}\KeywordTok{nothing} \NormalTok{happens]}
\end{Highlighting}
\end{Shaded}

\begin{itemize}
\tightlist
\item
  git does not track empty folders
\end{itemize}

\begin{Shaded}
\begin{Highlighting}[]
\NormalTok{$ }\KeywordTok{echo} \StringTok{'qwer'} \KeywordTok{>} \NormalTok{folder1/asdf.txt}
\NormalTok{$ }\KeywordTok{git} \NormalTok{add folder1}
\end{Highlighting}
\end{Shaded}

\begin{itemize}
\item
  nested git repositories are dangerous. don't do it.
\item
  To make git forget about a directory, simply:

\begin{Shaded}
\begin{Highlighting}[]
\NormalTok{$ }\KeywordTok{rm} \NormalTok{-rf .git}
\end{Highlighting}
\end{Shaded}

  \subsection{.git}\label{git}
\end{itemize}

\begin{Shaded}
\begin{Highlighting}[]
\NormalTok{$ }\KeywordTok{ls} \NormalTok{.git}

\KeywordTok{branches}
\KeywordTok{COMMIT_EDITMSG}
\KeywordTok{config}
\KeywordTok{description}
\KeywordTok{HEAD}
\KeywordTok{hooks}
\KeywordTok{index}
\KeywordTok{info}
\KeywordTok{logs}
\KeywordTok{objects}
\KeywordTok{refs}
\end{Highlighting}
\end{Shaded}

\begin{itemize}
\tightlist
\item
  contains many files, including a local instance of
  \textasciitilde{}/.config
\end{itemize}

\begin{Shaded}
\begin{Highlighting}[]
\NormalTok{$ }\KeywordTok{cat} \NormalTok{COMMIT_EDITMSG}


\CommentTok{# Please enter the commit message for your changes. Lines starting}
\CommentTok{# with '#' will be ignored, and an empty message aborts the commit.}
\CommentTok{# On branch master}
\CommentTok{# Changes to be committed:}
\CommentTok{#   modified:   git_workshop_notes.md}
\CommentTok{#   modified:   git_workshop_notes.pdf}
\CommentTok{#   modified:   git_workshop_notes.tex}
\CommentTok{#}
\CommentTok{# Changes not staged for commit:}
\CommentTok{#   deleted:    git_workshop_notestex}
\CommentTok{#   deleted:    git_workshop_notestex.aux}
\CommentTok{#   deleted:    git_workshop_notestex.log}
\CommentTok{#   deleted:    git_workshop_notestex.out}
\CommentTok{#   deleted:    git_workshop_notestex.pdf}
\CommentTok{#}
\CommentTok{# Untracked files:}
\CommentTok{#   .git_workshop_notes.md.swo}
\CommentTok{#   .git_workshop_notes.md.swp}
\CommentTok{#}
\end{Highlighting}
\end{Shaded}

\begin{itemize}
\tightlist
\item
  This is what will be shown upon commit; contains stuff from last
  stage.
\end{itemize}

\begin{Shaded}
\begin{Highlighting}[]
\NormalTok{$ }\KeywordTok{cat} \NormalTok{.git/HEAD}

\KeywordTok{ref}\NormalTok{: refs/heads/master}
\end{Highlighting}
\end{Shaded}

\begin{itemize}
\tightlist
\item
  To commit quickly: ```bash \$ git commit -am `quickly commit more
  notes without multiple commands or staging'
\end{itemize}

{[}master 1064b17{]} quickly commit more notes without multiple commands
or staging 9 files changed, 249 insertions(+), 1086 deletions(-) create
mode 100644 folder1/asdf.txt rewrite git\_workshop\_notes.pdf (64\%)
delete mode 100644 git\_workshop\_notestex delete mode 100644
git\_workshop\_notestex.aux delete mode 100644
git\_workshop\_notestex.log delete mode 100644
git\_workshop\_notestex.out delete mode 100644
git\_workshop\_notestex.pdf ```

If you committed a mistake and want to amend

\begin{Shaded}
\begin{Highlighting}[]
\NormalTok{$ }\KeywordTok{git} \NormalTok{commit --amend}
\CommentTok{# opens vi editor and allows editing of commit message, uses same hash and everything}
\NormalTok{$ }\KeywordTok{git} \NormalTok{log --oneline}

\KeywordTok{daa3b30} \NormalTok{quickly commit more notes without multiple commands or staging with slight changes}
\KeywordTok{904dcea} \NormalTok{deleted accidental files and added more notes}
\KeywordTok{0e0d499} \NormalTok{Start notes on this tutorial folder}
\end{Highlighting}
\end{Shaded}

\subsection{Demo}\label{demo}

\begin{Shaded}
\begin{Highlighting}[]
\NormalTok{$ }\KeywordTok{mkdir} \NormalTok{git_workshop_2}
\NormalTok{$ }\KeywordTok{cd} \NormalTok{git_workshop_2}
\NormalTok{$ }\KeywordTok{vim} \NormalTok{file1.txt}
\NormalTok{$ }\KeywordTok{cat} \NormalTok{file1.txt}

\KeywordTok{asdfl;kjasdfl}
\KeywordTok{asdfl;kasdfl}

\NormalTok{$ }\KeywordTok{git} \NormalTok{init}

\KeywordTok{Initialized} \NormalTok{empty Git repository in /home/alex/git_workshop_2/.git/}

\NormalTok{$ }\KeywordTok{ls}

\KeywordTok{file1.txt}

\NormalTok{$ }\KeywordTok{git} \NormalTok{add file1.txt}
\NormalTok{$ }\KeywordTok{git} \NormalTok{commit -m }\StringTok{'new'}

\NormalTok{[}\KeywordTok{master} \NormalTok{(root-commit) }\KeywordTok{6aaa158}\NormalTok{] new}
 \KeywordTok{1} \NormalTok{file changed, 3 insertions(+)}
 \KeywordTok{create} \NormalTok{mode 100644 file1.txt}

\NormalTok{$ }\KeywordTok{vim} \NormalTok{file1.txt}
\NormalTok{$ }\KeywordTok{cat} \NormalTok{file1.txt}

\KeywordTok{sdfl;kjasdfl}
\KeywordTok{asdfl;kasdfl}
\KeywordTok{fjdkfjdkfdk}

\KeywordTok{and} \NormalTok{now for something completely different}

\NormalTok{$ }\KeywordTok{git} \NormalTok{diff}

\KeywordTok{diff} \NormalTok{--git a/file1.txt b/file1.txt}
\KeywordTok{index} \NormalTok{9dea9ee..ab8d3ea 100644}
\KeywordTok{---} \NormalTok{a/file1.txt}
\KeywordTok{+++} \NormalTok{b/file1.txt}
\KeywordTok{@@} \NormalTok{-1,3 +1,5 @@}
 \KeywordTok{asdfl;kjasdfl}
 \KeywordTok{asdfl;kasdfl}
 \KeywordTok{fjdkfjdkfdk}
\KeywordTok{+}
\KeywordTok{+and} \NormalTok{now for something completely different}

\end{Highlighting}
\end{Shaded}

\hypertarget{refs}{}

\end{document}
